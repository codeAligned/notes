\documentclass{article}
\usepackage{nips_2018}

\usepackage[utf8]{inputenc} % allow utf-8 input
\usepackage[T1]{fontenc}    % use 8-bit T1 fonts
\usepackage{hyperref}       % hyperlinks
\usepackage{url}            % simple URL typesetting
\usepackage{booktabs}       % professional-quality tables
\usepackage{amsfonts}       % blackboard math symbols
\usepackage{nicefrac}       % compact symbols for 1/2, etc.
\usepackage{microtype}      % microtypography
\usepackage{amsmath}

\title{Probability Theory}

\author{
	Manik Bhandari\thanks{Use footnote for providing further
		information about author (webpage, alternative
		address)---\emph{not} for acknowledging funding agencies.} \\
	Department of Computer Science\\
	Indian Institute of Science\\
	Bangalore, India \\
	\texttt{mbbhandarimanik2@gmail.com} \\
}

\begin{document}
	% \nipsfinalcopy is no longer used
	
	\maketitle
	
	\begin{abstract}
		Notes of Probability Theory mainly from the book \textit{Introduction to Probability Theory by Hoel, Port, Stone.}
	\end{abstract}
	
	\section{Probability Spaces}
	
	\section{Discrete Random Variables}
	
	\section{Expectation of Discrete Random Variables}
	\begin{itemize}
		\item Let $n$ be the number of trials and $N_n(x_i)$ be the number of times you observed $x_i$, then for large n \[f(x_i) = \frac{N_n(x_i)}{n}.\]
		\item A random variable $X$ has \textit{finite} expectation if and only if \[\sum_{i=0}^{i=\infty}|x_i|f(x_i) < \infty, \] then the expectation is \[EX = \sum_{i=0}^{i=\infty}x_if(x_i).\]
				Otherwise $EX$ is undefined.
		\item Quick Expectations to remember: Binomial - $np$, Poisson - $\lambda$, Geometric - $\frac{1-p}{p}$.
		\item Expectation of a function (if it is finite): \[E\phi(X) = \sum_{x}\phi(x)f(x)\]
		\item Properties
			\subitem $E(c_1X_1 + c_2X_2 + ...) = c_1EX_1 + c_2EX_2 + ...$
			\subitem $|EX| \leq E|X|$
			\subitem $X \leq Y \implies EX \leq EY$
			\subitem If $P(|X| \leq M) = 1$ then $EX \leq M$ 
			\subitem if $X$ and $Y$ are independent, $E(XY) = (EX)(EY)$. The converse is not true.
			\subitem if $X$ is a non negative, integer valued random variable, $X$ has a finite expectation if and only if $\sum_{x=0}^{x=\infty}P(X \geq x)$ converges. Then $EX = \sum_{x=0}^{x=\infty}P(X \geq x)$.
	\end{itemize}
	
	
	
	\section*{References}
	
	References follow the acknowledgments. Use unnumbered first-level
	heading for the references. Any choice of citation style is acceptable
	as long as you are consistent. It is permissible to reduce the font
	size to \verb+small+ (9 point) when listing the references. {\bf
		Remember that you can use more than eight pages as long as the
		additional pages contain \emph{only} cited references.}
	\medskip
	
	\small
	
	[1] Alexander, J.A.\ \& Mozer, M.C.\ (1995) Template-based algorithms
	for connectionist rule extraction. In G.\ Tesauro, D.S.\ Touretzky and
	T.K.\ Leen (eds.), {\it Advances in Neural Information Processing
		Systems 7}, pp.\ 609--616. Cambridge, MA: MIT Press.
	
	[2] Bower, J.M.\ \& Beeman, D.\ (1995) {\it The Book of GENESIS:
		Exploring Realistic Neural Models with the GEneral NEural SImulation
		System.}  New York: TELOS/Springer--Verlag.
	
	[3] Hasselmo, M.E., Schnell, E.\ \& Barkai, E.\ (1995) Dynamics of
	learning and recall at excitatory recurrent synapses and cholinergic
	modulation in rat hippocampal region CA3. {\it Journal of
		Neuroscience} {\bf 15}(7):5249-5262.
	
\end{document}