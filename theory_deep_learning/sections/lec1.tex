\section{Introduction}
\textbf{Define} error of a classifier (aka \textit{true error}) as probability of it making a mistake given a random data point.
\[
    L_{D,f}(h) = \underset{x \sim D}P[h(x) \neq f(x)] = D(\{x: h(x) \neq f(x)\})
\]
where D is the distribution from where a data point x is drawn, $f$ is a known \textit{correct} function which always gives the correct labels to a data point. By this definition $D(A)$ is the probability of observing a random point $x$ from $A$.

\textbf{Define} \textit{training error} or \textit{emperical risk} as 
\[
    L_S(h) = \frac{|i \in [m]: h(x_i) \neq y_i|}{m}
\]
where $S$ is the training \textit{set} (it is actually a sequence since points can repeat and classifiers often take into account their order) of the form $\{(x_i, y_i)\}$. If you naively minimize this emperical risk then you are likely to overfit. To avoid it, you use some prior knowledge about the \textit{kind of classifier} that can possibly fit to the data and restrict your hypothesis search space to those types of classifiers.


This kind of restriction induces a \textit{bias} in the model (aka \textit{inductive bias}). In this setting, define 
\[
    h_S = ERM_h(S) \in \underset{h \in \mathcal{H}}{argmin}~L_S(h).
\]
This is a tradeoff -- choosing a restricted $\mathcal{H}$ can add too much bias but choosing a large $\mathcal{H}$ may lead to overfitting.

\paragraph{Finite hypothesis class} If we restrict $\mathcal{H}$ to have an upper bound on its size then $ERM_h$ will not overfit if we have \textit{large} training data (how large will depend on size of $\mathcal{H})$.

\paragraph{Realizability Assumption} There exists $h^* \in \mathcal{H}$ such that $L_{D,f}(h^*) = 0$ i.e. it never makes a mistake which means that $L_S(h^*) = 0$. Since this is the least possible error, this means that for every $ERM$ hypothesis $L_S(h_S) = 0$. We are however interested in true error of $h_S$ i.e. $L_{D,f}(h_S)$.

\paragraph{iid assumption} Assume that elements of $S$ are identically and independently distributed according to $D$ denoted by $S \sim D^m$.

Now, we would like to have an $h_S$ such that $L_{D,f}(h_S)$ is not too large. Let's say $h_S$ \textit{fails} if $L_{D,f}(h_S) > \epsilon$.

We want to upper bound the probability of sampling a training set that leads to a failure i.e $D^m({S: L_{D,f}(h_S) > \epsilon}).$ Define bad hypothesis as $\mathcal{H_B} = \{h \in \mathcal{H}\}: L_{D,f}(h_S) > \epsilon$ and misleading training sets as $M = \{S: \exists h \in \mathcal{H_B}, L_S(h) = 0\}$. So, all the training sets for which $h_S$ fails must be misleading (there can be other misleading sets also). So
\[
	\{S: L_{D,f}(h_S) > \epsilon \} \subseteq M = \bigcup_{h \in \mathcal{H_B}} \{S:  L_S(h) = 0\}.
\]
This means that 
\[
	D^m(\{S: L_{D,f}(h_S) > \epsilon \}) \leq D^m(M) = D^m(\bigcup_{h \in \mathcal{H_B}}\{S: L_S(h) = 0\}).
\]
Take union bound of RHS to get 
\[
	D^m(\{S: L_{D,f}(h_S) > \epsilon \}) \leq \sum_{h \in \mathcal{H_B}} D^m(\{S: L_S(h) = 0\}) = \sum_{h \in \mathcal{H_B}} \bigg(\prod_{i=1}^{m} D(\{x_i: h(x_i) = f(x_i)\})\bigg).
\]
and since $h \in \mathcal{H_B}$, $D(\{x_i: h(x_i) = f(x_i)\}) \leq 1 - \epsilon$. But each $x_i$ is iid over $D$ so $D^m(\{S: L_S(h) = 0 \leq (1-\epsilon)^m \leq e^{-\epsilon m}$. As $m$ goes large, the probability of finding a misleading set reduces. Therefore
\[
	D^m(\{S: L_{D,f}(h_S) > \epsilon \}) \leq |\mathcal{H_B}|e^{-\epsilon m} \leq |\mathcal{H}|e^{-\epsilon m}.
\]
Take log both sides to get
\[
\log D^m(\{S: L_{D,f}(h_S) > \epsilon \})  \leq \log |\mathcal{H}| - \epsilon m \implies m \leq \frac{\log( |\mathcal{H}|/\delta)}{\epsilon}
\]
where $\delta = D^m(\{S: L_{D,f}(h_S) > \epsilon \})$. This also implies that if $m$ is large enough i.e. $ m \geq \frac{\log( |\mathcal{H}|/\delta)}{\epsilon}$ then $L_{D,f}(h_S) \leq \epsilon$ with probability $1 - \delta$ of choosing the iid samples $S$. \\

So with the $ERM_h$ rule, your hypothesis will be \textit{probably} ($1-\delta$) \textit{approximately} ($\epsilon$) \textit{correct} (PAC). Note that the size $m$ does not depend upon the underlying distribution or labeling function.

\paragraph{PAC Learnability} A hypothesis class $\mathcal{H}$ is PAC learnable if $\exists~ m_{\mathcal{H}}: (0,1)^2 \to \mathbb{N}$ and a learning algorithm such that\\
For every $(\epsilon, \delta) \in (0,1)$, for every distribution $\mathcal{D}$ over $\mathcal{X}$ and for every labeling function $f: \mathcal{X} \to (0,1)$\\
If the realizability assumption holds over $\mathcal{H}, \mathcal{D}, f$\\
then running the algorithm on $m > m_{\mathcal{H}}(\epsilon, \delta)$ samples generated iid from $\mathcal{D}$ and labeled by $f$ gives a hypothesis $h$ such that \\
with probability at least $1-\delta$ over the choice of examples, $L_{\mathcal{D}, f}(h) \leq \epsilon$.

\paragraph{Sample complexity} $m_{\mathcal{H}}: (0,1)^2 \to \mathbb{N}$ defines the \textit{sample complexity} of learning $\mathcal{H}$ i.e. how many samples are required to get a PAC solution. Let it be the \textit{minimum function} that satisfies the criteria of PAC learnability.

\paragraph{Sample complexity of finite hypothesis class} Every finite hypothesis class is PAC learnable with sample complexity $ m \leq \lceil \frac{\log( |\mathcal{H}|/\delta)}{\epsilon} \rceil$

\paragraph{Removing realizability assumption} Assuming that such an $h^*$ exists such that \\
$\underset{x \sim \mathcal{D}}{P}[h^*(x) = f(x)] = 1$ is too strong. Not only might such an $h^*$ not exists, your features might not be discriminative enough. Instead assume that $\mathcal{D}$ is a joint distribution over domain points $\mathcal{X}$ and labels $\mathcal{Y}$.
Now, true error 
\[
	L_D(h) = \underset{(x,y~ \sim ~\mathcal{D})}{P}[h(x) \neq y] = \mathcal{D}(\{(x,y): h(x) \neq y\})
\]

\paragraph{Bayes optimal predictor} is the best labeling function defined as
\[
	f_{\mathcal{D}}(x) = [insert brackets] 1 if \mathbb{P}[y=1 | x] \geq 0.5
									0 o.w
\]

\paragraph{Agnostic PAC Learnability} A hypothesis class $\mathcal{H}$ is agnostic PAC learnable w.r.t. a set $\mathbb{Z}$ and a loss function $l: \mathcal{Z} \to \mathbb{R}_+$ if there exists a function $m_\mathcal{H}: (0,1)^2 \to \mathbb{N}$ and a learning algorithm such that\\
for \textit{every} $\epsilon, \delta \in (0,1)$ and for  \textit{every} $\mathcal{D}$ over $\mathcal{Z}$ when the algorithm is run $m \geq m_\mathcal{H}(\epsilon, \delta)$ samples iid from $\mathcal{D}$, the algo returns a hypothesis $h \in \mathcal{H}$ such that with probability $1 - \delta$ over the training samples
\[
	L_\mathcal{D}(h) = \underset{h' \in \mathcal{H}}{min} L_\mathcal{D}(h') + \epsilon
\]
where $L_\mathcal{D}(h) = \mathbb{E}_{z\sim\mathcal{D}}[l(h,z)]$.