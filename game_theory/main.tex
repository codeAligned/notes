\documentclass{article}
\usepackage[final]{nips_2018}

\usepackage[utf8]{inputenc} % allow utf-8 input
\usepackage[T1]{fontenc}    % use 8-bit T1 fonts
\usepackage{hyperref}       % hyperlinks
\usepackage{url}            % simple URL typesetting
\usepackage{booktabs}       % professional-quality tables
\usepackage{amsfonts}       % blackboard math symbols
\usepackage{nicefrac}       % compact symbols for 1/2, etc.
\usepackage{microtype}      % microtypography
\usepackage{amsmath}
\usepackage{amsthm}
\usepackage{xcolor}
\usepackage{wrapfig}

\newtheorem{theorem}{Theorem}

\title{Game Theory}

\author{
	Manik Bhandari\\
	Department of Computational Data Science\\
	Indian Institute of Science\\
	Bangalore, India \\
	\texttt{mbbhandarimanik2@gmail.com} \\
}


\begin{document}
	
	\maketitle
	
	\begin{abstract}
		Notes of Game Theory mainly from lectures at IISC.
	\end{abstract}
	
\section{Introduction and Examples}
The course can be divided into four parts. We start with Non cooperative Game Theory.

\paragraph{Game Theory vs Mechanism Design}
In Game Theory, you are given a game and you must analyze it. \textit{Analyzing} a game typically means finding and analyzing the \textit{Equilibrium}. An informal definition of equilibrium is when all the players of the game are \textit{happy} i.e. if they were to perform some other action, they would not be happy. 

In Mechanism Design, you are given a \textit{desirable behavior} of the players i.e. a \textit{Social Choice Function}. Your task, is to design a game that implements the SCF in the \textit{optimal case}. \textcolor{red}{Optimal case means equilibrium?}

\paragraph{Student Coordination Problem}
Consider two students, call them $P_1$ and $P_2$. They wish to meet and chat. They can either do that in IISC or on MG Road. These two actions are called \textit{strategies}. Let "going to IISC" be strategy $A$ and "going to MG Road" be strategy $B$. Then the \textit{strategy sets} of the two players are $S_1 = S_2 = \{A, B\}$. 

Assume that \textit{strategies are chosen simultaneously}. This is called a \textit{Simultaneous Moves Game}. A game like chess is not a simultaneous moves game. The fact that player must choose strategies means that this is a \textit{Strategic form Game}. \textcolor{red}{What other form of game can exist?}.

The \textit{utility} of a player is a function $u_i: S_1 \times S_2 \to \mathbb{R}$ (subscript $i$ for player). It maps a \textit{vector or profile} of strategies to a real number called \textit{utility \textcolor{red}{or payoff?}}. 

\begin{wraptable}{r}{5cm}
	\begin{tabular}{lcc}
		\toprule
		1/2 & A & B\\
		\midrule
		A & 100,100 & 0, 0 \\
		B & 0, 0    & 10, 10\\
		\bottomrule
	\end{tabular}
	\caption{Payoff matrix.}
\end{wraptable}

\paragraph{Assumption of Common Knowledge} This means that 
\begin{enumerate}
	\item Each player knows the payoff matrix.
	\item Each player knows 1.
	\item Each player knows 2.
	\item .
	\item .
\end{enumerate}
ad infinitum. 

\paragraph{Players are Rational and Intelligent} A \textit{rational player} maximizes the \textit{Expected value of his own payoff}. This is called \textit{Expected Utility Maximization}. An \textit{Intelligent} player is capable of computing the optimal strategy.

\paragraph{Equilibrium} In this example, there are two equilibria -- $(A,A)$ and $(B,B)$. The intuition is that if a player would choose some other action, he won't be as happy. \textcolor{red}{This is not clear}. This is an example of \textit{Pure Strategy Equilibrium} but another possibility is to have a \textit{probabilistic strategy} leading to \textit{Mixed Strategy Equilibrium}. A strategy will then be a vector of probabilities of choosing each action.

In this case the mixed strategy equilibrium is $(1/11, 10/11), (1/11, 10/11)$ i.e. players can choose to be in MG road most of the time and still be pretty happy.

\paragraph{Cake cutting problem} This is an example of Mechanism Design. Suppose you and your brother want to eat a cake and your mom (The mechanism designer) is to decided how to divide the cake. The SCF is not just fair division, it is to make everyone involved \textit{happy}.

If she cuts the cake, you might feel that your brother will pick the bigger part leading to dissatisfaction. Consider this strategy: you cut the cake but your brother picks the first piece. Since you cut the cake and there is \textit{common knowledge}, you are absolutely sure that it is exactly in half so you're satisfied. Since he picks the first, he might pick what he thinks is the bigger part and he'll be happy. Possible extensions for this could be to $n$ cakes and $m$ players.

\paragraph{Braess Paradox} Adding more links should help but doesn't. Why?

\paragraph{Vickrey Auction} Auctions are essential. Google has a few slots to display for ads and many companies would like to place their ad in the slot. If the user clicks on an ad, the company pays Google for the click. How to decide the price for obtaining the slot? Best way is to bid for it.  

\textit{Winner Determination:} The winner is decided by whoever bids the highest.

\textit{Pricing:} In \textit{First price auctions} the winning bid would be the price of the slot. But Vickrey introduced \textit{Second price auctions} where the second highest bid or the highest non-winning bid is the price of the slot.

This extracts the exact \textit{validation} of each player i.e. the maximum willingness to buy the item. This also satisfies \textit{allocative efficiency} i.e. the item goes to the person who values it the most. It also has the \textit{DSIC property (Dominant Strategy Incentive Compatibility)} which means that you can safely ignore other players and always bid your true validation. (There also exists others like BIC -- Bayesian Incentive Compatibility) There is also no complexity of computing the strategy.

This has led to VCG (Vickrey Clarke Groves) mechanism

\paragraph{Cooperative Game Theory} This is the form of GT when people form a coalition or a group for their benefit.

\paragraph{Divide the dollar problem} Suppose you have 300\$ and you want to divide it amongst three people such that \textit{at least two should agree} on the division. Let's say $P_1$ proposes $(150, 150, 0)$. This is also a simultaneous moves game. We say that coalition $\{1,2\}$ is formed. But $P_3$ will not sit quietly. He proposes $(200, 0, 100)$. Coalition $\{ 1,3\}$ is formed since both $P_1$ and $P_3$ have increased their utility. But now $P_2$ will not sit quietly, he proposes $(0, 100, 200)$ and thus Coalition $\{2, 3\}$ is formed. This is a highly unstable situation where players will keep cycling.

Variations of this could be similar game but "a leader and one other should agree" or "all three should agree". The dynamics will completely change. 

\section{Utility Theory}
\paragraph{Every game of chess ends in either a win for white, win for black, or draw.}
Simple proof idea: think of the game of chess as a tree with children being the reachable board positions from parent. Start from the leaf and build up.

\paragraph{preference relation} A player wants his \textit{best possible outcome}. To measure this we want a relation between all possible outcomes. \textcolor{red}{An order?}. This relation is \textit{preference relation}.
It is a binary relation over the set of possible outcomes $O$ i.e. $x \geq_{i} y$ implies that player $i$ prefers outcome $x$ over $y$ or is indifferent between the two.

\paragraph{Assumptions about the preference relation}
\begin{enumerate}
    \item It is complete i.e. for any pair of outcomes $x, y$ in $O$ either $x \geq_{i} y$ or $x \leq_{i} y$  or both.
    \item It is reflexive i.e. $x \geq_{i} x$ for all $x \in O$.
    \item It is transitive i.e. for any $x,y,z \in O$ if $x \geq_{i} y$ and $y \geq_{i} z$ then $x \geq_{i} z$.
\end{enumerate}

\paragraph{Utility function} is a mapping from $O \to \mathbb{R}$ such that if $x \geq_{i} y$ then $u(x) \geq_{i} u(y)$. Essentially it is an association of preference with a real number.
Note that for any monotonically increasing function $v:\mathbb{R} \to \mathbb{R}$, $v \cdot u = v(u(x))$ is also a utility function for the same preference relation. So a utility function $u$ is only \textit{ordinal} i.e the real number is not a measure of intensity of a player's preference.

\section{Strategic form games a.k.a. Normal form}
\paragraph{Definition} A Strategic Form Game is an ordered triple $G = (N, (S_i)_{i \in N}, (u_i)_{i \in N})$. $N$ is the \textit{finite} set of players $N = {1,2,..,n}$, $S_i$ is the set of strategies of player $i$. Denote the set of all possible strategy vectors as $S = S_1 \times S_2 ... \times S_n$. And $u_i: S \to \mathbb{R}$ maps a vector of strategies $s = (s_i)_{i \in N}$ to the payoff of the $i^{th}$ player. 

Note the strategy set of a player need not be finite. If it is finite, we call it a \textit{finite game}.

\paragraph{Notation} Denote $X$ as the cross product $\times_{i \in N}X_i$. Then $X_{-i} = \times_{j\neq i}X_j$ i.e. an element of $X_{-i}$ is an n-1 dimensional vector $x_{i-1} = (x_i, ... x_{i-1}, x_{i+1}... x_n)$.

\paragraph{Domination} The strategy $s_i$ of a player is dominated by $t_i$ if for \textit{each} strategy of other players $s_{-i} \in S_{-i}$ 
\[
    u_i(s_i, s_{-i}) < u_i(t_i, s_{-i})
\]
So $t_i$ strictly dominates $s_i$ or $s_i$ is strictly dominated by $t_i$.

\paragraph{Assumptions} A rational player will never choose a strictly dominated strategy and all players in the game are rational. Also assume rationality is common knowledge.

\paragraph{Strictly dominant strategy} of a player is one which strictly dominates all other strategies of the player.

\section{Strategic form games a.k.a. Normal form}
\paragraph{Definition} A Strategic Form Game is an ordered triple $G = (N, (S_i)_{i \in N}, (u_i)_{i \in N})$. $N$ is the \textit{finite} set of players $N = {1,2,..,n}$, $S_i$ is the set of strategies of player $i$. Denote the set of all possible strategy vectors as $S = S_1 \times S_2 ... \times S_n$. And $u_i: S \to \mathbb{R}$ maps a vector of strategies $s = (s_i)_{i \in N}$ to the payoff of the $i^{th}$ player. 

Note the strategy set of a player need not be finite. If it is finite, we call it a \textit{finite game}.

\paragraph{Notation} Denote $X$ as the cross product $\times_{i \in N}X_i$. Then $X_{-i} = \times_{j\neq i}X_j$ i.e. an element of $X_{-i}$ is an n-1 dimensional vector $x_{i-1} = (x_i, ... x_{i-1}, x_{i+1}... x_n)$.

\paragraph{Domination} The strategy $s_i$ of a player is dominated by $t_i$ if for \textit{each} strategy of other players $s_{-i} \in S_{-i}$ 
\[
    u_i(s_i, s_{-i}) < u_i(t_i, s_{-i})
\]
So $t_i$ strictly dominates $s_i$ or $s_i$ is strictly dominated by $t_i$.

\paragraph{Assumptions} A rational player will never choose a strictly dominated strategy and all players in the game are rational. Also assume rationality is common knowledge.

\paragraph{Strictly dominated strategy} of a player is one which strictly dominates all other strategies of the player.

\paragraph{Iterated elimination of dominant strategies} Since all players are rational and this is common knowledge, and rational players never choose a dominated strategy, we can safely eliminate such a strategy from a player's strategy set. Iteratively doing this is called Iterated elimination of dominated strategies. When this process yields only \textit{one} strategy per player, the strategy vector obtained is called \textit{solution} of game.

Therefore, solution always exists if each player has a \textit{strictly dominant strategy}.

\paragraph{Order of elimination} does not matter when we iteratively eliminate strictly dominated strategies. You'll always get the same set of strategies. \textcolor{red}{prove this}. However, for eliminating weakly dominated strategies, you might get the different results.

\paragraph{Weakly dominated strategy} $s_i$ is weakly dominated by $t_i$ if 
\begin{enumerate}
	\item for every strategy $s_{-i} \in S_{-i}$ of other players, $u_i(s_i, s_{-i}) \leq u_i(t_i, s_{-i})$ \textit{and}
	\item there exists at least one strategy $t_{-i} \in S_{-i}$ of other players such that $u_i(s_i, t_{-i}) < u_i(t_i, t_{-i})$.
\end{enumerate}

Henceforth, dominated = weakly dominated. A rational player will never choose a dominated strategy. A \textit{rational strategy} is one which is reached by iterative elimination of weakly dominated strategies a.k.a. \textit{rationalizability}. 

Elimination of a player's dominant strategy does not require the knowledge of other players' payoffs. However, in rationalizability when we eliminate $s_i$ of player $i$ after $s_j$ of player $j$, player $i$ assume that player $j$ will never play $s_j$.

\paragraph{Second price auction as Strategic Form Game} Let $N$ be the set of players, $S_i = [0, \infty]$ is the set of bids that the player can make, utility for a given strategy vector $b$ is 
\[
u_i(b) = 
\begin{cases}
0 \quad \quad \quad \quad \quad \quad \quad \text{if} \quad b_i \neq \underset{j}{max}b_j\\
\frac{v_i - \underset{i \neq j}{max}b_j}{|\{k: b_k = \underset{j}{max}b_j\}|} \quad \quad \text{if} \quad b_i = \underset{j}{max}b_j
\end{cases}
\]
Second case arises from the fact that $k$ people can give the same maximum bid in which case, the winner will be decided by a fair lottery. Note that, in this case the second highest bid will also be the same as $v_i$ and so the profit will be 0.

In this game, the strategy $b_i = v_i$ weakly dominates all other strategies.

\paragraph{Definition of Nash Equilibrium} A strategy vector $s^* = (s_1^*, s_2^*, ..., s_n^*)$ is a Nash Equilibrium if for \textit{each} player $i \in N$ and for each strategy $s_i \in S_i$, $u_i(s*) \geq u_i(s, s_{-i}^*)$. And the corresponding payoff vector $u(s^*)$ is the \textit{equilibrium payoff }corresponding to the Nash eqbm $s^*$.

\paragraph{Profitable Deviation} of player $i$ from a strategy vector $s$, is a strategy $\hat{s_i} \in S_i$ such that $u_i(\hat{s_i}, s_{-i}) > u_i(s_i)$.

A Nash eqbm is a strategy vector from which no player has a profitable (unilateral) deviation.





\section{Lecture 4}
\section{Solutions other than Nash Equilibrium}
We will find solutions that \textit{rational} players might not choose but \textit{pessimistic} players might.

A pessimistic player might want to maximize the minimum payoff he can get, disregarding whether other players are rational or not. i.e. 
\[
	\underline{v_i} := \underset{s_i \in S_i}{max}~\underset{t_{-i} \in S_{-i}}{min}~u_i(s_i, t_{-i})
\]
This quantity $\underline{v_i}$ is the \textit{maxmin value} of player $i$ or the \textit{security} of player $i$. The strategy $s_i^*$ that guarantees this value is the \textit{maxmin strategy}. It is also known as \textit{no regret strategy} since a player is never worse off with any other strategy.

Since this strategy $s_i^*$ is the max, it satisfies $\underset{t_{-i} \in S_{-i}}{min}~u_i(s_i^*, t_{-i}) \geq \underset{t_{-i} \in S_{-i}}{min}~u_i(s_i, t_{-i})~ \forall s_i \in S_i$. Since this is true for all $s_i \in S_i$, it must be true for the maxmin strategy also. Therefore, $\underset{t_{-i} \in S_{-i}}{min}~u_i(s_i^*, t_{-i}) \geq \underline{v_i} ~ \forall t_{-i} \in S_{-i}$ . This is another way of saying player $i$ is never worse off than $\underline{v_i}$ no matter what the other players play.

\textcolor{red}{[side note]} You have to be careful if max and min are not defined, let's say when the strategy sets are infinite. Then you must replace max and min by sup and inf respectively. Even then, they might not exist.

\paragraph{Dominant strategy and maxmin strategy} The dominant strategy is also the maxmin strategy and is also the nash equilibrium strategy.
\textcolor{red}{proof?. Reconfirm this. It sounds wrong.}

\paragraph{Strictly dominant and maxmin} If each player has a strictly dominant strategy $s_i^*$ then $(s_1^*, ..., s_n^*)$ is a unique \textit{equilibrium }of the game and the vector is also the vector of \textit{maxmin }strategies.

\paragraph{nash equilibrium and security} Every Nash Equilibrium $\sigma^*$ of a strategic form game satisfies $u_i(\sigma^*) \geq \underline{v_i}$ for every player $i$.

This is because for every player $i$, $u_i(s_i, s_{-i}) \geq \underset{s_{-i}}{min}~u_i(s_i, s_{-i})$. Taking max over $s_i$ on both sides, you get $\underset{s_i}{max} ~u_i(s_i, s_{-i}) \geq \underset{s_i}{max} ~\underset{s_{-i}}{min}~u_i(s_i, s_{-i})$ which proves the theorem.
	
	
\end{document}