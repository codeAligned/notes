\documentclass{article}
\usepackage[final]{nips_2018}

\usepackage[utf8]{inputenc} % allow utf-8 input
\usepackage[T1]{fontenc}    % use 8-bit T1 fonts
\usepackage{hyperref}       % hyperlinks
\usepackage{url}            % simple URL typesetting
\usepackage{booktabs}       % professional-quality tables
\usepackage{amsfonts}       % blackboard math symbols
\usepackage{nicefrac}       % compact symbols for 1/2, etc.
\usepackage{microtype}      % microtypography
\usepackage{amsmath}
\usepackage{amsthm}
\usepackage{xcolor}
\usepackage{wrapfig}

\newtheorem{theorem}{Theorem}

\title{Game Theory}

\author{
	Manik Bhandari\\
	Department of Computational Data Science\\
	Indian Institute of Science\\
	Bangalore, India \\
	\texttt{manikb@iisc.ac.in} \\
}


\begin{document}
	
	\maketitle
	
	\begin{abstract}
		My attempt at solving the Problem Sets for Game Theory at IISC 2019.
	\end{abstract}
	
\section{Introduction}
\subsection{warm-up}
\begin{enumerate}
	\item A game in my understanding is an \textit{interaction} between agents or players. They take some \textit{actions} to achieve some goal (usually the maximization of their expected utility). These actions can be simultaneous or sequential \textcolor{red}{or mixed?}, actions can be one step or multi step, their actions and action histories may/may not be visible to others, the players themselves may/may not be rational and/or intelligent. 
	
	In a non-cooperative game, players will not form a coalition with one another and will not play \textit{together} whereas in cooperative GT, players form coalitions and decide to play a defined strategy (to increase their utility).
	\item In game theory, we are given a game and we analyze various types of equilibria in the game whereas in mechanism design we are given a social choice function that we want to implement and we design a game that implements it in the optimal case. 
	
	e.g. Fair utilization of computational resources of a lab. Given a set of resources at your disposal and players wanting to use them (their utilities can be based on the proximity to any upcoming deadlines that they are targeting), how do yo fairly divide resources among all the members.
	\item Intelligence assumes that players are capable of computing the \textit{optimal strategy} (for rational players this strategy is maximizing their expected utility). Rationality assumes that players will play to maximize their expected utility.
	\item In the matching pennies problem, Player 1 wins on a match and Player 2 wins on a mismatch. Since the probabilities of matching and not matching are equally likely $(\frac{2}{4}, \frac{2}{4})$, the expected utility of both the players should be zero. \textcolor{red}{very doubtful on this.}
\end{enumerate}

\subsection{Workhorse}
\begin{enumerate}
	\item \textcolor{red}{Need to get the book}
	\item The strategic form game here $G = (N, (S_i)_{i \in N}, (u_i)_{i \in N})$ consists of $N = \{1,2, ..., n\}$, $S_i = \{1,2, ..., m\}$, the strategy vector $s \in S$ will be $(x_1, x_2, ..., x_n)$ where each $x_i \in \{1,2, ..., m\}$.
	\[
		u_i(x_1, x_2, ..., x_n) = 
		\begin{cases}
			\frac{1}{|\{k: x_k = argmin~(|x_k - (2/3)\bar{x}|)\}|} \quad \text{if} \quad x_i = argmin~(|x_i - (2/3)\bar{x}|)\\
			0 \quad \text{otherwise}
		\end{cases}
	\]
	Can we get a PSNE analysis for this?
	\item \textcolor{red}{Pigou network. need to study.}
	\item \textcolor{red}{Not sure} First person cuts in half, second cuts each half in one-sixth. Third person picks first, then second, then first.
\end{enumerate}
	
	
\end{document}