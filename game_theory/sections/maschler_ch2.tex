\section{Utility Theory}
\paragraph{Every game of chess ends in either a win for white, win for black, or draw.}
Simple proof idea: think of the game of chess as a tree with children being the reachable board positions from parent. Start from the leaf and build up.

\paragraph{preference relation} A player wants his \textit{best possible outcome}. To measure this we want a relation between all possible outcomes. \textcolor{red}{An order?}. This relation is \textit{preference relation}.
It is a binary relation over the set of possible outcomes $O$ i.e. $x \geq_{i} y$ implies that player $i$ prefers outcome $x$ over $y$ or is indifferent between the two.

\paragraph{Assumptions about the preference relation}
\begin{enumerate}
    \item It is complete i.e. for any pair of outcomes $x, y$ in $O$ either $x \geq_{i} y$ or $x \leq_{i} y$  or both.
    \item It is reflexive i.e. $x \geq_{i} x$ for all $x \in O$.
    \item It is transitive i.e. for any $x,y,z \in O$ if $x \geq_{i} y$ and $y \geq_{i} z$ then $x \geq_{i} z$.
\end{enumerate}

\paragraph{Utility function} is a mapping from $O \to \mathbb{R}$ such that if $x \geq_{i} y$ then $u(x) \geq_{i} u(y)$. Essentially it is an association of preference with a real number.
Note that for any monotonically increasing function $v:\mathbb{R} \to \mathbb{R}$, $v \cdot u = v(u(x))$ is also a utility function for the same preference relation. So a utility function $u$ is only \textit{ordinal} i.e the real number is not a measure of intensity of a player's preference.


