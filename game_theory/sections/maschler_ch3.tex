\section{Strategic form games a.k.a. Normal form}
\paragraph{Definition} A Strategic Form Game is an ordered triple $G = (N, (S_i)_{i \in N}, (u_i)_{i \in N})$. $N$ is the \textit{finite} set of players $N = {1,2,..,n}$, $S_i$ is the set of strategies of player $i$. Denote the set of all possible strategy vectors as $S = S_1 \times S_2 ... \times S_n$. And $u_i: S \to \mathbb{R}$ maps a vector of strategies $s = (s_i)_{i \in N}$ to the payoff of the $i^{th}$ player. 

Note the strategy set of a player need not be finite. If it is finite, we call it a \textit{finite game}.

\paragraph{Notation} Denote $X$ as the cross product $\times_{i \in N}X_i$. Then $X_{-i} = \times_{j\neq i}X_j$ i.e. an element of $X_{-i}$ is an n-1 dimensional vector $x_{i-1} = (x_i, ... x_{i-1}, x_{i+1}... x_n)$.

\paragraph{Domination} The strategy $s_i$ of a player is dominated by $t_i$ if for \textit{each} strategy of other players $s_{-i} \in S_{-i}$ 
\[
    u_i(s_i, s_{-i}) < u_i(t_i, s_{-i})
\]
So $t_i$ strictly dominates $s_i$ or $s_i$ is strictly dominated by $t_i$.

\paragraph{Assumptions} A rational player will never choose a strictly dominated strategy and all players in the game are rational. Also assume rationality is common knowledge.

\paragraph{Strictly dominated strategy} of a player is one which strictly dominates all other strategies of the player.

\paragraph{Iterated elimination of dominant strategies} Since all players are rational and this is common knowledge, and rational players never choose a dominated strategy, we can safely eliminate such a strategy from a player's strategy set. Iteratively doing this is called Iterated elimination of dominated strategies. When this process yields only \textit{one} strategy per player, the strategy vector obtained is called \textit{solution} of game.

Therefore, solution always exists if each player has a \textit{strictly dominant strategy}.

\paragraph{Order of elimination} does not matter when we iteratively eliminate strictly dominated strategies. You'll always get the same set of strategies. \textcolor{red}{prove this}. However, for eliminating weakly dominated strategies, you might get the different results.

\paragraph{Weakly dominated strategy} $s_i$ is weakly dominated by $t_i$ if 
\begin{enumerate}
	\item for every strategy $s_{-i} \in S_{-i}$ of other players, $u_i(s_i, s_{-i}) \leq u_i(t_i, s_{-i})$ \textit{and}
	\item there exists at least one strategy $t_{-i} \in S_{-i}$ of other players such that $u_i(s_i, t_{-i}) < u_i(t_i, t_{-i})$.
\end{enumerate}

Henceforth, dominated = weakly dominated. A rational player will never choose a dominated strategy. A \textit{rational strategy} is one which is reached by iterative elimination of weakly dominated strategies a.k.a. \textit{rationalizability}. 

Elimination of a player's dominant strategy does not require the knowledge of other players' payoffs. However, in rationalizability when we eliminate $s_i$ of player $i$ after $s_j$ of player $j$, player $i$ assume that player $j$ will never play $s_j$.

\paragraph{Second price auction as Strategic Form Game} Let $N$ be the set of players, $S_i = [0, \infty]$ is the set of bids that the player can make, utility for a given strategy vector $b$ is 
\[
u_i(b) = 
\begin{cases}
0 \quad \quad \quad \quad \quad \quad \quad \text{if} \quad b_i \neq \underset{j}{max}b_j\\
\frac{v_i - \underset{i \neq j}{max}b_j}{|\{k: b_k = \underset{j}{max}b_j\}|} \quad \quad \text{if} \quad b_i = \underset{j}{max}b_j
\end{cases}
\]
Second case arises from the fact that $k$ people can give the same maximum bid in which case, the winner will be decided by a fair lottery. Note that, in this case the second highest bid will also be the same as $v_i$ and so the profit will be 0.

In this game, the strategy $b_i = v_i$ weakly dominates all other strategies.

\paragraph{Definition of Nash Equilibrium} A strategy vector $s^* = (s_1^*, s_2^*, ..., s_n^*)$ is a Nash Equilibrium if for \textit{each} player $i \in N$ and for each strategy $s_i \in S_i$, $u_i(s*) \geq u_i(s, s_{-i}^*)$. And the corresponding payoff vector $u(s^*)$ is the \textit{equilibrium payoff }corresponding to the Nash eqbm $s^*$.

\paragraph{Profitable Deviation} of player $i$ from a strategy vector $s$, is a strategy $\hat{s_i} \in S_i$ such that $u_i(\hat{s_i}, s_{-i}) > u_i(s_i)$.

A Nash eqbm is a strategy vector from which no player has a profitable (unilateral) deviation.




